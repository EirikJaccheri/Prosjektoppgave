\documentclass{article}
\usepackage[utf8]{inputenc}
\usepackage{amsmath,mathtools}
\usepackage{biblatex}
\addbibresource{prosjektoppgave.bib}

\title{Prosjektoppgave}
\author{Eirik Jaccheri}
\date{Autumn 2022}
\usepackage{graphicx}
\usepackage{layout}
\setlength{\voffset}{-0.75in}
\setlength{\headsep}{5pt}
\setlength{\textheight}{700pt} 


%Spørsmål jeg har til senere:
%Hvordan kan jeg se at E og B felt blir maksimalt på ulike steder


\begin{document}

\maketitle

\section{Derivation of interaction hamiltonian}

\subsection{Quantization of vector potetntial in a cavity}
We follow the derivation of \cite{QuantizationEMCavities}
%explain why we can quantize the modes?
\begin{equation}
    \textbf{A} = i \sum_{\vec{q},s} \sqrt{\frac{\hbar}{\epsilon \epsilon_0 V w_q}} e^{i(q_x x + q_y y )} \hat{e}_{\vec{q},s}\left(a_{\vec{q},s} + a_{-\vec{q}, s}\right)
    \label{vector potential cavity}
\end{equation}


\subsection{Paramagnetic coupling}
Energy of particle experiencing a electromagnetic field is given by the Hamiltonian
\begin{equation}
    H_A = \frac{1}{2m} \sum_{\sigma} \int dr \psi_{\sigma}^{\dagger}(r) \left( \frac{\hbar}{i} \nabla - q \textbf{A} \right)^2 \psi_{\sigma}(r) 
\end{equation}
The terms linear in the vector potential $\textbf{A}$, define the paramagnetic interaction hamiltonian
\begin{equation}
    H_{int} = \sum_{\sigma} \int dr  \psi_{\sigma}^{\dagger}(r) \frac{q\hbar}{2mi}\left(  \nabla \cdot \textbf{A}  -  \textbf{A} \cdot \nabla \right) \psi_{\sigma}(r) 
\end{equation}
By partial integration it can be rewritten
\begin{equation}
    H_{int} = \sum_{\sigma} \int dr \frac{q\hbar}{2mi} \textbf{A} \cdot \left[  (\nabla  \psi_{\sigma}^{\dagger}(r)) \psi_{\sigma}(r)   -   \psi_{\sigma}^{\dagger}(r) (\nabla \psi_{\sigma}(r))  \right]
\end{equation}
The expression in the square bracket give the paramagnetic current density. From classical mechanics, the current density is defined by the equation
%hvor kommer dette fra?
\begin{equation}
    \delta H = - q \int dr J \cdot \delta A
\end{equation}
Using this equation one can identify the current density
\begin{equation}
    J_i = \frac{\hbar}{2mi} (\psi_{\sigma}^{\dagger}(r) (\nabla_i \psi_{\sigma}(r)) - (\nabla_i  \psi_{\sigma}^{\dagger}(r)) \psi_{\sigma}(r))
\end{equation}
restricting the positions of the atoms to discrete positions $\vec{n}$ and discretizing the nabla operator gives
%tror jeg diskretiserer feil
\begin{align}
    J_i &= \frac{\hbar}{2mi} (C_{\vec{n},\sigma}^{\dagger} \frac{C_{\vec{n + 1_i},\sigma} - C_{\vec{n},\sigma}}{a}  - \frac{C_{\vec{n + 1_i},\sigma}^{\dagger} - C_{\vec{n},\sigma}^{\dagger}}{a} C_{\vec{n},\sigma}) \hat{e}_i\\
      &= \frac{\hbar}{2mai} (C_{\vec{n},\sigma}^{\dagger} C_{\vec{n + 1_i},\sigma} - C_{\vec{n + 1_i},\sigma}^{\dagger} C_{\vec{n},\sigma}^{\dagger})\hat{e}_i
\end{align}
wich gives the current 
\begin{equation}
    \vec{J} = \frac{\hbar}{2mai} \sum_{i=x,y}\sum_{\sigma} (C_{\vec{n},\sigma}^{\dagger} C_{\vec{n + 1_i},\sigma} - C_{\vec{n + 1_i},\sigma}^{\dagger} C_{\vec{n},\sigma})\hat{e}_i
    \label{electric current}
\end{equation}
%feil prefaktor og fortegn...
%schlawin summer over n både i hamiltonian og current
with interaction hamiltonian 
\begin{equation}
    H_{int} = \sum_{\vec{n}} \vec{J} \cdot \textbf{A}((\vec{r}_{\vec{n + 1_i}} + \vec{r}_{\vec{n}})/2 )
    \label{interaction hamiltonian general}
\end{equation}
\subsection{Coupling of cavity to metal}
Using the expression for the vector potential~\ref{vector potential cavity} and electric current~\ref{electric current} in the interaction hamiltonian~\ref{interaction hamiltonian general}
\begin{align*}
    \begin{split}
    H_{int} &= \sum_{\vec{n}} \left(\frac{iaet}{\hbar} \sum_{i=x,y}\sum_{\sigma} (C_{\vec{n},\sigma}^{\dagger} C_{\vec{n + 1_i},\sigma} - C_{\vec{n + 1_i},\sigma}^{\dagger} C_{\vec{n},\sigma})\hat{e}_i\right) \cdot \\     
            &\left(i \sum_{\vec{q},s} \sqrt{\frac{\hbar}{\epsilon \epsilon_0 V w_q}} e^{i\left(q_x \left(x_{\vec{n + 1_i}} + x_{\vec{n}} \right) + q_y \left(y_{\vec{n + 1_i}} + y_{\vec{n}} \right)\right)/2} \hat{e}_{\vec{q},s}\left(a_{\vec{q},s} + a_{-\vec{q}, s}\right)\right)
    \end{split}\\
    %
    \begin{split}
    H_{int} &= \frac{-aet}{\sqrt{\hbar \epsilon \epsilon_0 V} } \sum_{\sigma,\vec{q},s} \left(a_{\vec{q},s} + a_{-\vec{q}, s}\right) \frac{1}{\sqrt{w_q}} \\ 
            &  \sum_{i=x,y} \sum_{\vec{n}} \hat{e}_{\vec{q},s,i} (C_{\vec{n},\sigma}^{\dagger} C_{\vec{n + 1_i},\sigma} - C_{\vec{n + 1_i},\sigma}^{\dagger} C_{\vec{n},\sigma}) e^{i\left(q_x \left(x_{\vec{n + 1_i}} + x_{\vec{n}} \right) + q_y \left(y_{\vec{n + 1_i}} + y_{\vec{n}} \right)\right)/2}
    \end{split}
     %
     \label{interaction hamiltonian step 1}\textbf{}
\end{align*}
Introducing fourier transformed operators $C_{k,\sigma} , C_{k,\sigma}^{\dagger}$
\begin{align*}
    \begin{split}
    H_{int} &= \frac{-aet}{\sqrt{\hbar \epsilon \epsilon_0 V} } \sum_{\sigma,\vec{q},s} \left(a_{\vec{q},s} + a_{-\vec{q}, s}\right) \frac{1}{\sqrt{w_q}} \\ 
            & \sum_{k,k'}   C_{\vec{k},\sigma}^{\dagger} C_{\vec{k'},\sigma} \sum_{i=x,y} \left[\frac{1}{N}\sum_{\vec{n}} \hat{e}_{\vec{q},s,i} ( e^{i\left(k \cdot r_{\vec{n}} - k' \cdot r_{\vec{n+1_i}} \right)} - e^{i\left(k \cdot r_{\vec{n+1_i}} - k' \cdot r_{\vec{n}} \right)}) e^{i\left(q_x \left(x_{\vec{n + 1_i}} + x_{\vec{n}} \right) + q_y \left(y_{\vec{n + 1_i}} + y_{\vec{n}} \right)\right)/2}\right]
    \end{split}
     %
    \label{interaction hamiltonian step 2}
\end{align*}
Performing the summation in the square brackets setting $i = x$ and using $x_{\vec{n}},y_{\vec{n}} = a n_x, a n_y$
\begin{align}
    [...] &= \frac{1}{N}\sum_{n_x, n_y} \hat{e}_{\vec{q},s,x} ( e^{i a\left((k_x - k_x') n_x - k_x' + (k_y - k_y') n_y \right)} -  e^{i a\left((k_x - k_x') n_x + k_x + (k_y - k_y') n_y \right)}) e^{i a \left(q_x \left(n_x + 1/2\right) + q_y (n_y + 1/2)\right)}\\
          &= \frac{1}{N}\sum_{n_x, n_y} \hat{e}_{\vec{q},s,x} e^{i a\left((k_x - k_x' + q_x) n_x + (k_y - k_y' + q_y) n_y \right)} e^{ia q_x / 2} ( e^{- i a k_x' } -  e^{i a k_x })\\
          &= a \hat{e}_{\vec{q},s,x} \delta_{k_x',k_x + q_x} \delta_{k_y',k_y + q_y}  ( e^{- i a (k_x + q_x /2) } -  e^{i a (k_x + q_x/2)  })\\
          &=   -2 i a \hat{e}_{\vec{q},s,x} \delta_{\vec{k}',\vec{k} + \vec{q}} \sin(k_x + q_x /2)\\
\end{align}
where in the last line we have used $\sum_n e^{ian(k - k')} = N a \delta_{k,k'}$. Performing the analogus calculation for $i = y$, equation~\ref{interaction hamiltonian step 2} becomes
\begin{align}
\begin{split}
    H_{int} &=  \sum_{\vec{k},\sigma,\vec{q},s} \frac{g_{\vec{k},s}^{(\vec{q})}}{\sqrt{N}} \left(a_{\vec{q},s} + a_{-\vec{q}, s}\right) C_{\vec{k},\sigma}^{\dagger} C_{\vec{k + \vec{q}},\sigma} 
    \end{split}
     %
\end{align}
with 
\begin{equation}
   g_{\vec{k},s}^{(\vec{q})} = i t \sqrt{\frac{4 e^2}{\hbar \epsilon \epsilon_0 L_z}} \frac{\hat{e}_{\vec{q},s,x} sin(k_x + q_x /2) + \hat{e}_{\vec{q},s,y} \sin(k_x + q_x /2)}{\sqrt{c \sqrt{\vec{q}^2}}}
   %Får ikke konstantene til å gå opp
\end{equation}
where $c = c_0 / \sqrt{\epsilon}$, we have used that $V = N a^2 L_z$

\printbibliography

\end{document}
