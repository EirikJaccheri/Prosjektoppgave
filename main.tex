\documentclass{article}
\usepackage[utf8]{inputenc}
\usepackage{amsmath,mathtools}
\usepackage{biblatex}
\usepackage{cleveref}
\usepackage{todonotes}
\usepackage{physics}
\addbibresource{prosjektoppgave.bib}

\title{Prosjektoppgave}
\author{Eirik Jaccheri}
\date{Autumn 2022}
\usepackage{graphicx}
\usepackage{layout}
\setlength{\voffset}{-0.75in}
\setlength{\headsep}{5pt}
\setlength{\textheight}{700pt} 



%Spørsmål jeg har til senere:
%hva er greia med conservation of momentum?
%hva er greia med anticrossings?
%Hva betyr "resolve broken inversion symmetry in SC?
%Hvordan kan man se på directional dependence of superconducting gap?




%besvarte spm
%Er vår hamiltonian på form 4.26 s167 i atlas and simon? (har det noe å si at vi har 2 forskjellige typer partikler?
    %svar: tror det er trivielt å legge til to ekstra kordinater. Men har ikke gjort det eksplisitt
%Hva er relasjon mellom det vi gjør og mean field theory i Simon kapittel 6.1?
    %svar: jeg blandet effective og mean field theory. I kap 3 skal jeg gjøre effective field theory som i 6.1
%Hva er lattice cites i normalmetallet? Er det sånn tight binding greie?
    %svar: er tught binding med antagelse om nesten fri. Henning mener at å anta tight binding og lattice sites kanskje er unødig.
%Hvordan kan jeg se at E og B felt blir maksimalt på ulike steder
    %svar: dette er trivielt på grunn av cos og sin i vektor potensiale?

\begin{document}

\maketitle

\section{Derivation of interaction hamiltonian}

\subsection{Quantization of vector potetntial in a cavity}
%TODO: figur av cavity
%Burde jeg legge til maxwells likninger for å forklare transversal etc?
%Burde introdusere orthogonale modefunksjoner for at kvantisering skal gi mening?
We follow the derivation in \cite{QuantizationEMCavities} for the quantization of the electromagnetic modes inside a rectangular cavity with side lengths $L_x$, $L_y$ and $L_z$. The walls in the $z$-direction are assumed to be perfectly conducting, leading to the boundary conditions $\textbf{E}|_{tan} = B|_{norm} = 0$ at $z=0$ and $z = L_z$. Furthermore we assume that $L_x, L_y >> L_z$ allowing us to impose periodic boundary conditions in the $x$ and $y$-directions. These boundary conditions are satisfied by the vector potential
\begin{equation}
    \textbf{A} = \sum_{\textbf{q},i} \sqrt{\frac{\hbar}{2 \epsilon \epsilon_0 V w_q}} \left(b_{\textbf{q}, i} u_{\textbf{q},i} + b_{\textbf{q}, i}^{*} u_{\textbf{q},i}^{\dagger}\right)\;,
    \label{classical vector potential}
\end{equation}
with the mode functions
\begin{align}
    u_{\textbf{q}, x} = u_{\textbf{q}, y} &= \sqrt{\frac{2}{V}} e^{iq_x x} e^{iq_y y} i \sin{q_z z}\nonumber\\
                     u_{\textbf{q}, z} &= \sqrt{\frac{2}{V}} e^{iq_x x} e^{iq_y y} \cos{q_z z} \label{mode functions}\;,
\end{align}

%hvorfor må b-ene v-være akkurat slik=
where $V$ is the volume of the cavity, $\omega_{\textbf{q}}$ is the frequency of the electromagnetic wave and $b_{\textbf{q}, i}$ is the field amplitude in the $\hat{e}_i$ direction. The wave vector can take the values $\textbf{q} = (2\pi q_x / L_x, 2\pi q_y / L_y, \pi q_z / L_z )$, with $q_x, q_y = 0, \pm 1, \pm 2...$ and $q_z = 0, 1, 2...$. 

Because of the gauge freedom in the vector potential  $\textbf{A}$, the field amplitudes are not independent.  Choosing the Coulomb gauge, the vector potential must be transversal $\textbf{A} \cdot \textbf{q} = 0$ . It is convenient to rotate to a new basis $\bar{e}_{\textbf{q}, s}, s \in (1,2,3)$, where the new $z$-axis (labeled by $3$) is parallel to $\textbf{q}$. The new basis vectors are given by $\bar{e}_{\textbf{q}, s} = \sum_i O^{\textbf{q}}_{si} \hat{e}_i $, with
\begin{equation}
    O^{\textbf{q}} = \begin{pmatrix} 
    \cos{\theta}\cos{\phi} & \cos{\theta}\sin{\phi} & - \sin{\theta} \\ 
    -\sin{\phi}            & \cos{\phi}             & 0 \\
    \sin{\phi}\cos{\phi}   & \sin{\theta}\sin{\phi} & \cos{\theta}
    \end{pmatrix}\;,
\end{equation}
where the angles $\theta$, $\phi$ depend on $\textbf{q}$. The field amplitudes in this basis become $a_{\textbf{q},s} = \sum_i O^{\textbf{q}}_{s,i} b_{\textbf{q}, i}$ and $a_{\textbf{q},s}^{*} = \sum_i O^{\textbf{q}}_{s,i} b_{\textbf{q}, i}^{*}$. Written in this basis, the vector potential becomes

\begin{equation}
\textbf{A} = \sum_{\textbf{q}, s} \sqrt{\frac{\hbar}{2 \epsilon \epsilon_0 V w_q}} \left(a_{\textbf{q},s} \bar{u}_{\textbf{q},s} + a_{\textbf{q},s}^{*} \bar{u}_{\textbf{q},s}^{\dagger}\right)\;,
\end{equation}
where we have defined the new mode functions $\bar{u}_{\textbf{q},s} = \sum_i e_{q,i} O^{\textbf{q}}_{s,i} u_{\textbf{q}, i}$. Using the facts that the field amplitudes are now independent and that the mode functions are orthogonal $\int_c \bar{u}_{\textbf{q},s} \cdot \bar{u}_{\textbf{q}',s'} = \delta_{s,s'}\delta_{\textbf{q},\textbf{q}'} $, the field amplitudes can be quantized using the canonical quantization procedure~\cite{Altland}
\begin{equation}
\textbf{A} = \sum_{\textbf{q}, s} \sqrt{\frac{\hbar}{2 \epsilon \epsilon_0 V w_q}} \left(a_{\textbf{q},s} \bar{u}_{\textbf{q},s} + a_{\textbf{q},s}^{\dagger} \bar{u}_{\textbf{q},s}^{\dagger}\right)\;, 
\end{equation}
with the comutation relations
\begin{equation}
    \left[a_{\textbf{q},s}, a_{\textbf{q'},s'}\right] = \left[a_{\textbf{q},s}^{\dagger}, a_{\textbf{q'},s'}^{\dagger}\right] = 0, \left[a_{\textbf{q},s}, a_{\textbf{q'},s'}^{\dagger}\right] = i \hbar \delta_{\textbf{q}, \textbf{q}'} \delta_{s,s'}\;.
\end{equation}
Using \cref{mode functions}, taking $q_z = 1$, evaluating the vector potential at $\pi z_0 / L = \pi / 2$ and the fact that $u_{\textbf{q},s}^{\dagger} = u_{\textbf{\bar{q}},s}̃$ ($\textbf{\bar{q}} = (-q_x, -q_y, q_z)$ ) gives the final expression 
\begin{equation}
    \textbf{A} = i \sum_{\textbf{q},s} \sqrt{\frac{\hbar}{\epsilon \epsilon_0 V w_q}} e^{i(q_x x + q_y y )} \hat{e}_{\textbf{q},s}\left(a_{\textbf{q},s} + a_{-\textbf{q}, s}^{\dagger}\right) \;.
    \label{vector potential cavity}
\end{equation}
%hvorfor kan vi nå kvantisere?


\subsection{Paramagnetic coupling}
Energy of particle experiencing a electromagnetic field is given by the Hamiltonian
\begin{equation}
    H_A = \frac{1}{2m} \sum_{\sigma} \int dr \psi_{\sigma}^{\dagger}(r) \left( \frac{\hbar}{i} \nabla - q \textbf{A} \right)^2 \psi_{\sigma}(r)\;.
\end{equation}
The terms linear in the vector potential $\textbf{A}$, define the paramagnetic interaction hamiltonian 
\begin{equation}
    H_{int} = \sum_{\sigma} \int dr  \psi_{\sigma}^{\dagger}(r) \frac{q\hbar}{2mi}\left(  \nabla \cdot \textbf{A}  -  \textbf{A} \cdot \nabla \right) \psi_{\sigma}(r) \;.
\end{equation}
By partial integration it can be rewritten
\begin{equation}
    H_{int} = \sum_{\sigma} \int dr \frac{q\hbar}{2mi} \textbf{A} \cdot \left[  (\nabla  \psi_{\sigma}^{\dagger}(r)) \psi_{\sigma}(r)   -   \psi_{\sigma}^{\dagger}(r) (\nabla \psi_{\sigma}(r))  \right] \;.
\end{equation}
From classical mechanics, the electrical current density is defined by the equation
%hvor kommer dette fra?
\begin{equation}
    \delta H = - \int dr J \cdot \delta A \;.
    \label{current classical mech}
\end{equation}
Using \cref{current classical mech} one can identify the paramagnetic current density
\begin{equation}
    J_i = \frac{q \hbar}{2mi} (\psi_{\sigma}^{\dagger}(r) (\nabla_i \psi_{\sigma}(r)) - (\nabla_i  \psi_{\sigma}^{\dagger}(r)) \psi_{\sigma}(r))\;.
\end{equation}
Restricting the positions of the atoms to discrete positions $\textbf{n}$ and discretizing the nabla operator gives
\begin{align}
    J_i &= \frac{q \hbar}{2mi} (C_{\textbf{n},\sigma}^{\dagger} \frac{C_{\textbf{n} + \textbf{1}_i,\sigma} - C_{\textbf{n},\sigma}}{a}  - \frac{C_{\textbf{n} + \textbf{1}_i,\sigma}^{\dagger} - C_{\textbf{n},\sigma}^{\dagger}}{a} C_{\textbf{n},\sigma}) \hat{e}_i\nonumber\\
      &= \frac{q \hbar}{2mai} (C_{\textbf{n},\sigma}^{\dagger} C_{\textbf{n} + \textbf{1}_i,\sigma} - C_{\textbf{n} + \textbf{1}_i,\sigma}^{\dagger} C_{\textbf{n},\sigma}^{\dagger})\hat{e}_i\;.\label{current component} 
\end{align}
Using \cref{current component} and the expression for the effective mass of a electron in a tight binding model $m^* = \frac{\hbar^2}{t a^2}$, the current vector becomes
\begin{equation}
    \textbf{J} = \frac{a q t}{2i\hbar} \sum_{i=x,y}\sum_{\sigma} \left( C_{\textbf{n},\sigma}^{\dagger} C_{\textbf{n} + \textbf{1}_i,\sigma} - C_{\textbf{n} + \textbf{1}_i,\sigma}^{\dagger} C_{\textbf{n},\sigma}\right)\hat{e}_i\;.
    \label{electric current}
\end{equation}
The paramagnetic interaction hamiltonian becomes 
\begin{equation}
    H_{int} = \sum_{\textbf{n}} \textbf{J} \cdot \textbf{A}((\textbf{r}_{\textbf{n} + \textbf{1}_i} + \textbf{r}_{\textbf{n}})/2 )\;.
    \label{interaction hamiltonian general}
\end{equation}
\subsection{Coupling of cavity to metal}
Using the expression for the vector potential~\cref{vector potential cavity} and electric current~\cref{electric current} in the paramagnetic interaction hamiltonian~\cref{interaction hamiltonian general}, gives the interaction Hamiltonian
\begin{align*}
    \begin{split}
    H_{int} &= \sum_{\textbf{n}} \left(\frac{aet}{2i\hbar} \sum_{i=x,y}\sum_{\sigma} (C_{\textbf{n},\sigma}^{\dagger} C_{\textbf{n} + \textbf{1}_i,\sigma} - C_{\textbf{n} + \textbf{1}_i,\sigma}^{\dagger} C_{\textbf{n},\sigma})\hat{e}_i\right) \cdot \\     
            &\left(i \sum_{\textbf{q},s} \sqrt{\frac{\hbar}{\epsilon \epsilon_0 V w_q}} e^{i\left(q_x \left(x_{\textbf{n} + \textbf{1}_i} + x_{\textbf{n}} \right) + q_y \left(y_{\textbf{n} + \textbf{1}_i} + y_{\textbf{n}} \right)\right)/2} \hat{e}_{\textbf{q},s}\left(a_{\textbf{q},s} + a_{-\textbf{q}, s}^{\dagger}\right)\right)
    \end{split}\\
    %
    \begin{split}
            &= \frac{aet}{2\sqrt{\hbar \epsilon \epsilon_0 V} } \sum_{\sigma,\textbf{q},s} \left(a_{\textbf{q},s} + a_{-\textbf{q}, s}^{\dagger}\right) \frac{1}{\sqrt{w_q}} \\ 
            &  \sum_{i=x,y} \sum_{\textbf{n}} \hat{e}_{\textbf{q},s,i} (C_{\textbf{n},\sigma}^{\dagger} C_{\textbf{n} + \textbf{1}_i,\sigma} - C_{\textbf{n} + \textbf{1}_i,\sigma}^{\dagger} C_{\textbf{n},\sigma}) e^{i\left(q_x \left(x_{\textbf{n} + \textbf{1}_i} + x_{\textbf{n}} \right) + q_y \left(y_{\textbf{n} + \textbf{1}_i} + y_{\textbf{n}} \right)\right)/2}\;.
    \end{split}
     %
     \label{interaction hamiltonian step 1}\textbf{}
\end{align*}
This Hamiltonian describes the interaction between the photons in the cavity and the electrons in the metal. Introducing the Fourier transformed operator
\begin{equation}
C_{\textbf{k},\sigma} = \frac{1}{\sqrt{N}} \sum_{\textbf{n}} C_{\textbf{n},\sigma} e^{-i\textbf{r}_{\textbf{n}} \cdot \textbf{k}}\;,
\label{fourier electron operator}
\end{equation}
gives
\begin{align*}
    \begin{split}
    H_{int} &= \frac{aet}{2\sqrt{\hbar \epsilon \epsilon_0 V} } \sum_{\sigma,\textbf{q},s} \left(a_{\textbf{q},s} + a_{-\textbf{q}, s}^{\dagger}\right) \frac{1}{\sqrt{w_q}} \\ 
            & \sum_{k,k'}   C_{\textbf{k'},\sigma}^{\dagger} C_{\textbf{k},\sigma} \sum_{i=x,y} \left[\frac{1}{N}\sum_{\textbf{n}} \hat{e}_{\textbf{q},s,i} ( e^{i\left(k \cdot r_{\textbf{n}+\textbf{1}_i} - k' \cdot r_{\textbf{n}} \right)} - e^{i\left(k \cdot r_{\textbf{n}} - k' \cdot r_{\textbf{n}+\textbf{1}_i}\right)}) e^{i\left(q_x \left(x_{\textbf{n} + \textbf{1}_i} + x_{\textbf{n}} \right) + q_y \left(y_{\textbf{n} + \textbf{1}_i} + y_{\textbf{n}} \right)\right)/2}\right]\;.
    \end{split}
     %
    \label{interaction hamiltonian step 2}
\end{align*}
Performing the summation in the square brackets setting $i = x$ and using $x_{\textbf{n}},y_{\textbf{n}} = a n_x, a n_y$
\begin{align*}
    [...] &= \frac{1}{N}\sum_{n_x, n_y} \hat{e}_{\textbf{q},s,x} ( e^{i a\left((k_x - k_x') n_x + k_x + (k_y - k_y') n_y \right)} -  e^{i a\left((k_x - k_x') n_x - k_x' + (k_y - k_y') n_y \right)}) e^{i a \left(q_x \left(n_x + 1/2\right) + q_y (n_y + 1/2)\right)}\\
          &= \frac{1}{N}\sum_{n_x, n_y} \hat{e}_{\textbf{q},s,x} e^{i a\left((k_x - k_x' + q_x) n_x + (k_y - k_y' + q_y) n_y \right)} e^{ia q_x / 2} ( e^{i a k_x } -  e^{-i a k_x' })\\
          &= a \hat{e}_{\textbf{q},s,x} \delta_{k_x',k_x + q_x} \delta_{k_y',k_y + q_y}  ( e^{i a (k_x + q_x /2) } -  e^{-i a (k_x + q_x/2)  })\\
          &= 2 i a \hat{e}_{\textbf{q},s,x} \delta_{\textbf{k}',\textbf{k} + \textbf{q}} \sin(k_x + q_x /2)\\
\end{align*}
where in the last line we have used $\sum_n e^{ian(k - k')} = N a \delta_{k,k'}$. Performing the analogous calculation for $i = y$, equation~\cref{interaction hamiltonian step 2} becomes
\begin{align}
    H_{int} &=  \sum_{\textbf{k},\sigma,\textbf{q},s} \frac{g_{\textbf{k},s}^{(\textbf{q})}}{\sqrt{N}} \left(a_{\textbf{q},s} + a_{-\textbf{q}, s}^{\dagger}\right) C_{\textbf{k} + \textbf{q},\sigma}^{\dagger} C_{\textbf{k},\sigma} \;,
\end{align}
with 
\begin{equation}
   g_{\textbf{k},s}^{(\textbf{q})} = i \tilde{g}_0 \frac{\hat{e}_{\textbf{q},s,x} \sin{a(k_x + q_x /2)} + \hat{e}_{\textbf{q},s,y} \sin{a(k_y + q_y /2)}}{\left(1 + \left(\frac{c\textbf{q}}{\omega_0}\right)^2\right)^{1/4}}\;.
   \label{g constant}
\end{equation}
The constant $\tilde{g}_0 = t\sqrt{\frac{4 \alpha}{\sqrt{\epsilon}}}$ where $\alpha$ equals the fine structure constant. In the derivation of \cref{g constant} we have redefined $\textbf{q} = (2\pi q_x / L_x, 2\pi q_y / L_y) $, and used that  $c = c_0 / \sqrt{\epsilon}$, $V = N a^2 L_z$, $q_z = 1$,  and $\omega_0 = c\pi / L_z$.


\section{Effective electron theory}
\todo{Må utfylle om matsubara formalisme. diskutere sym/anti-sym grensebetingelse}
In thermal equilibrium, much of the information about a physical system is given by the partition function 
\begin{equation}
    \mathcal{Z} = \Tr{e^{-\left(\hat{H} - \mu \hat{N}\right)}} = \sum_n \bra{n} e^{-\left(\hat{H} - \mu \hat{N}\right)} \ket{n} \;.
    \label{partion function}
\end{equation}
Where $\hat{H}$ is the hamiltonian of the system, $\hat{N}$ is the number operator and the sum goes over all possible states of the system. For problems with a large number or infinite degrees of freedom, it is convenient to rewrite the partition function as a path integral~\cite{Altland}
\begin{equation}
    Z = \int_{\Psi_0}^{\pm \Psi_0} D(\Psi^{\dagger},\Psi) e^{-S/\hbar}\;,
    \label{general path integral}
\end{equation}
with $S = \int_0^{\beta}d\tau \Psi^{\dagger}\partial_{\Tau}\Psi + H(\Psi^{\dagger},\Psi)$. For bosonic fields, the $\Psi$ and $\Psi^{\dagger}$ variables are complex numbers and the upper integration limit is given by $+\Psi$. For fermionic fields the variables are Grassman numbers, and the upper integration limit is $-\Psi_0$.

In the present problem of a two dimensional metall in a rectangular cavity, the Hamiltonian depends on the bosonic photon fields $a$ and $a^{\dagger}$ and the fermionic electron fields $C$ and $C^{\dagger}$. The partition function is then given by the path integral
\begin{equation}
    \mathcal{Z} = \int D(C^{\dagger}, C) \int D(a^{\dagger}, a) e^{-S[C^{\dagger},C,a^{\dagger},a] / \hbar}\;,
    \label{path integral}
\end{equation}
with the action
\begin{align}
    S[C^{\dagger},C,a^{\dagger},a] &= \int_0^{\beta}\bigg[d\tau \sum_{\textbf{k},\sigma}\left( C_{\textbf{k},\sigma}^{\dagger} \partial_{\tau} C_{\textbf{k},\sigma} + \left(\epsilon_{\textbf{k}} - \mu\right) C_{\textbf{k},\sigma}^{\dagger} C_{\svec{k},\sigma}\right) \nonumber\\ 
                                  &+ \sum_{\textbf{q},s}\left( a_{\textbf{q},s}^{\dagger} \partial_{\tau} a_{\textbf{q},s} +q \hbar w_{\textbf{q}} a_{\textbf{q},s}^{\dagger}a_{\textbf{q},s}\right) \label{action} \\
                                  &+ \sum_{\textbf{k},\sigma,\textbf{q},s} \frac{g_{\textbf{k},s}^{(\textbf{q})}}{\sqrt{N}} \left(a_{\textbf{q},s} + a_{-\textbf{q}, s}^{\dagger}\right) C_{\textbf{k} + \textbf{q},\sigma}^{\dagger} C_{\textbf{k},\sigma}\bigg]\nonumber \;.
    %\label{action}
\end{align}
Where the $C_{\textbf{k},\sigma}^{\dagger}$ and $C_{\textbf{k},\sigma}$ are Grassman numbers and $a_{\textbf{q},s}^{\dagger}$ and $a_{\textbf{q},s}$ are complex numbers. It is convenient to introduce the Fourier decompositions
\begin{equation}
C_{\textbf{k},\sigma} = \frac{1}{\sqrt{\beta}}\sum_{\omega_n} C_{k,\sigma} e^{-i \omega_n \tau}\;\text{and}\; C_{k,\sigma} = \frac{1}{\sqrt{\beta}} \int_0^{\beta} d\tau C_{\textbf{k},\sigma}(\tau) e^{i \omega_n \tau}\;,
\label{fourier C}
\end{equation}
where we have defined the $3$-vector $k = (-\omega_n,\textbf{k})$, with $\omega_n = (2n + 1)\pi / \beta, n \in  \mathbb{Z}$. Similarly, for the photon operators
\begin{equation}
a_{\textbf{q},s} = \frac{1}{\sqrt{\beta}}\sum_{\Omega_n} a_{q,s} e^{-i \Omega_n \tau}\;\text{and}\; a_{q,s} = \frac{1}{\sqrt{\beta}} \int_0^{\beta} d\tau a_{\textbf{q},s}(\tau) e^{i \Omega_n \tau}\;,
\label{fourier a}
\end{equation}
with $q = (-\Omega_n,\textbf{q})$ and $\Omega_n = 2n\pi / \beta, n \in  \mathbb{Z}$. Using the Fourier decompositions~\cref{fourier C} and~\cref{fourier a} in the expression for the action~\cref{action} gives

\begin{align}
    S[C^{\dagger},C,a,a^{\dagger}] &= \sum_{k,\sigma} C^{\dagger}_{k,\sigma}\left(-i\hbar\omega_n - \mu + \epsilon_{\textbf{k}}\right) C_{k,\sigma} \nonumber \\
                                   &+ \sum_{q,s} a^{\dagger}_{q,s} \left(-i\hbar \Omega_n + \hbar \omega_{\textbf{q}}\right) a_{q,s} \label{matsubara action} \\
                                   &+ \sum_{k,\sigma,q,s} \frac{g_{\textbf{k},s}^{(\textbf{q})}}{\sqrt{\beta N}} \left(a_{q,s} + a_{-q, s}^{\dagger}\right) C^{\dagger}_{k + q,\sigma}C_{k,\sigma}\;.\nonumber
\end{align}
To get the effective field theory of the electrons in the metal, one must integrate over the photon degrees of freedom $a_{q,s}$ and $a_{q,s}^{\dagger}$. To make the integration possible, the $a$-dependent part of~\cref{matsubara action} has to be rewritten as a sum of bilinear terms
\begin{align}
     &\sum_{q,s} a^{\dagger}_{q,s} \left(-i\hbar\Omega_n + \hbar \omega_{\textbf{q}}\right) a_{q,s} + \sum_{k,\sigma,q,s} \frac{g_{\textbf{k},s}^{(\textbf{q})}}{\sqrt{\beta N}} \left(a_{q,s} + a_{-q, s}^{\dagger}\right) C^{\dagger}_{k + q,\sigma}C_{k,\sigma}\nonumber \\
     %
     &= \sum_{q,s} \left(-i\hbar\Omega_n + \hbar \omega_{\textbf{q}}\right)  \left(a^{\dagger}_{q,s} + \frac{\sum_{k,\sigma'} \frac{g_{\textbf{k},s}^{(\textbf{q})}}{\sqrt{\beta N}} C^{\dagger}_{k + q,\sigma'}C_{k,\sigma'} }{\left(-i\hbar\Omega_n + \hbar \omega_{\textbf{q}}\right)} \right) \label{rewriting action}\\
     %
     & \left(a_{q,s} + \frac{\sum_{k',\sigma} \frac{g_{\textbf{k'},s}^{(-\textbf{q})}}{\sqrt{\beta N}} C^{\dagger}_{k' - q,\sigma}C_{k,\sigma} }{\left(-i\Omega_{n} + \hbar \omega_{\textbf{q}}\right)}\right)
     %
     -\sum_{q,s,k,k',\sigma,\sigma'} \frac{g_{\textbf{k},s}^{(\textbf{q})} g_{\textbf{k'},s}^{(-\textbf{q})}  }{\beta N\left(-i\hbar\Omega_n + \hbar \omega_{\textbf{q}}\right)}
      C^{\dagger}_{k + q,\sigma'}C_{k,\sigma'} C^{\dagger}_{k' - q,\sigma}C_{k',\sigma}\nonumber \\
     %
     &:= \sum_{\textbf{q},s,n} \left(-i\hbar\Omega_n + \hbar \omega_{\textbf{q}}\right)  \bar{a}^{\dagger}_{q,s}\bar{a}_{q,s} + S^{e-e}_{eff}\;.\nonumber
\end{align}
In the last line we defined the shifted coordinates  $\bar{a}^{\dagger}_{q,s}$ and $\bar{a}_{q,s}$, and introduced the effective action
\begin{equation}
    S^{e-e}_{eff} = -\sum_{q,k,k',\sigma,\sigma'} \frac{1}{2 \beta} V_{k k' q}
      C^{\dagger}_{k + q,\sigma'}C_{k,\sigma'} C^{\dagger}_{k' - q,\sigma}C_{k',\sigma}\;,\nonumber \\
    \label{effective hamiltonian}
\end{equation}
with
\begin{equation}
V_{k k' q} = \frac{2}{N} \sum_s \frac{g_{\textbf{k},s}^{(\textbf{q})} g_{\textbf{k'},s}^{(-\textbf{q})}  }{\left(-i\hbar\Omega_n + \hbar \omega_{\textbf{q}}\right)}\;.
\end{equation}
Using~\cref{rewriting action}, the expression for the action~\cref{matsubara action} becomes
\begin{equation}
     S[C^{\dagger},C,\bar{a},\bar{a}^{\dagger}] = \sum_{q,s} \left(-i\hbar\Omega_n + \hbar \omega_{q}\right)  \bar{a}^{\dagger}_{q,s}\bar{a}_{q,s} + \sum_{k,\sigma} \left(-i\hbar\omega_n - \mu + \epsilon_{\textbf{k}}\right) C^{\dagger}_{k,\sigma} C_{k,\sigma} +  S^{e-e}_{eff} \;.
     \label{action shifted coordinates}
\end{equation}
Since the change of variables to $\bar{a}$ and $\bar{a}^{\dagger}$ amounted to a $a$-independent shift of $a$ and $a^{\dagger}$, the functional integration measure in \cref{path integral} remains unchanged $D(a,a^{\dagger}) = D(\bar{a},\bar{a}^{\dagger})$. 
The $\bar{a}$-dependent part of the integral can now be computed using the Gaussian integral formula~\cite{Altland}
\begin{equation}
    \int D(v^{\dagger},v) e^{\sum_{i,j,n} v_i^{\dagger} A_{i,j}^{n} v_j} = \frac{1}{\det{A}} \;.
    \label{gaussian integral}
\end{equation}
The result of the ${a}$-integration is a multiplicative constant in the partition function. Since expectation values consist of taking derivatives of the logarithm of the partition function, this constant can be neglected. Finally, the effective partion function for the electrons becomes
\begin{equation}
    \mathcal{Z}_{eff} = \int D(C^{\dagger}, C) e^{-S_{eff}[C^{\dagger},C] / \hbar}\;,
    \label{effective partion function}
\end{equation} 
with the effetive action
\begin{equation}
    S_{eff}[C^{\dagger},C] = \sum_{k,\sigma} \epsilon_k C^{\dagger}_{k,\sigma} C_{k,\sigma} +  S^{e-e}_{eff} \;,
    \label{effective action}
\end{equation}
where we have defined $\epsilon_k = \left(-i\hbar\omega_n - \mu + \epsilon_{\textbf{k}}\right)$.
\section{Hubbard-Stratonovich transformation}
The action in equation~\cref{effective action} involves four fermion fields, and can therefore not be computed exactly. We therefore proceed by constructing a mean field theory using the Hubbard-Stratonovich(HS) transformation in the cooper channel~\cite{Altland}. Starting with a path integral with a general four field action 
\begin{equation}
    Z = \int D(C^{\dagger}, C) e^{-\frac{1}{2\beta}C^{\dagger}_{\alpha} C^{\dagger}_{\alpha'} V_{\alpha \alpha' \beta \beta'} C_{\beta} C_{\beta'} / \hbar \;,
    \label{general four field path integral}
\end{equation}
where we use the Einstein summation convention, and each $\alpha_i$ is a set of quantum numbers. In order to perform the HS-transformation we need the inverse of the potential $V_{\alpha_1 \alpha_2 \alpha_3 \alpha_4}$. The fermionic commutation relations of the coherent states, lead to a unconventional inverse relation. In order to see this, we define the representation matrices $V_{\alpha_1 \alpha_2 \alpha_3 \alpha_4}$ and $V^{-1}_{\alpha_1 \alpha_2 \alpha_3 \alpha_4}$ in terms of the abstract operators $\hat{V}$ and $\hat{V}^{-1}$
\begin{align}
V_{\alpha_1 \alpha_2 \alpha_3 \alpha_4} &= \bra{\alpha_1 \alpha_2}\hat{V}\ket{\alpha_3 \alpha_4}\nonumber \\
V^{-1}_{\alpha_1 \alpha_2 \alpha_3 \alpha_4} &= \bra{\alpha_1 \alpha_2}\hat{V}^{-1} \ket{\alpha_3 \alpha_4} \label{V representation} \;.
\end{align}
Using this definition 
\begin{equation}
    V_{\alpha_1 \alpha_2 \beta \beta'} V^{-1}_{\beta' \beta \alpha_3 \alpha_4} = \bra{\alpha_1 \alpha_2}\hat{V}\ket{\beta \beta'}\bra{\beta' \beta}\hat{V}^{-1} \ket{\alpha_3 \alpha_4}  = \delta_{\alpha_1 \alpha_4} \delta_{\alpha_2 \alpha_3} - \delta_{\alpha_1 \alpha_3} \delta_{\alpha_2 \alpha_4} 
    \label{inverse relation}
\end{equation}
Where we have used the identity $\sum_{\beta} \ket{\beta}{\bra{\beta}} = 1$ and the anti-commutation relations between the coherent state creation/anihilation operators $C^{\dagger}_{\alpha}$ and ${C_{\alpha'}}$. Introdusing the new bosonic fields $\bar{\Delta}$ and $\Delta$, using the identity
\begin{equation} 
    1 = \int D(\bar{\Delta},\Delta) e^{\beta \bar{\Delta}_{\alpha \alpha'} V^{-1}_{\alpha \alpha' \beta \beta'} \Delta_{\beta \beta'} / \hbar }\;,
\end{equation}
and rescaling the fields
\begin{align}
    \bar{\Delta}_{\alpha \alpha'} &\rightarrow \bar{\Delta}_{\alpha \alpha'} + \frac{1}{2\beta}  C^{\dagger}_{\gamma} C^{\dagger}_{\gamma'} V_{\gamma,\gamma' \alpha' \alpha}\nonumber\\
         \Delta_{\beta \beta'}    &\rightarrow \Delta_{\beta \beta'} + \frac{1}{2\beta} V_{\beta' \beta \gamma \gamma'} C_{\gamma} C_{\gamma'}\label{rescaling hs-fields}\;,
\end{align}
allows us to rewrite the action in~\cref{general four field path integral} 
\begin{align}
       S &= -\frac{1}{2\beta}C^{\dagger}_{\alpha} C^{\dagger}_{\alpha'} V_{\alpha \alpha' \beta \beta'} C_{\beta} C_{\beta'} \nonumber\\
       %
       &\rightarrow \beta \left(\bar{\Delta}_{\alpha \alpha'} + \frac{1}{2 \beta} C^{\dagger}_{\gamma} C^{\dagger}_{\gamma'} V_{\gamma,\gamma' \alpha' \alpha}\right) V^{-1}_{\alpha \alpha' \beta \beta'} 
       %
       \left(\Delta_{\beta \beta'} + \frac{1}{2 \beta} V_{\beta' \beta \gamma \gamma'} C_{\gamma} C_{\gamma'}\right) \nonumber \\
       %
       &- \frac{1}{2\beta}C^{\dagger}_{\alpha} C^{\dagger}_{\alpha'} V_{\alpha \alpha' \beta \beta'} C_{\beta} C_{\beta'} \label{hs-transformation}\\
       %
       &= \beta \bar{\Delta}_{\alpha \alpha'} V^{-1}_{\alpha \alpha' \beta \beta'} \Delta_{\beta \beta'} + \frac{1}{2}\left( C^{\dagger}_{\gamma} C^{\dagger}_{\gamma'} V_{\gamma,\gamma' \alpha' \alpha}  V^{-1}_{\alpha \alpha' \beta \beta'} \Delta_{\beta \beta'}+ \bar{\Delta}_{\alpha \alpha'} V^{-1}_{\alpha \alpha' \beta \beta'} V_{\beta' \beta \gamma \gamma'} C_{\gamma} C_{\gamma'}\right) \nonumber \\
       %
       &+ \frac{1}{4\beta} C^{\dagger}_{\gamma} C^{\dagger}_{\gamma'} V_{\gamma,\gamma' \alpha' \alpha}  V^{-1}_{\alpha \alpha' \beta \beta'} V_{\beta' \beta \gamma \gamma'} C_{\gamma} C_{\gamma'}  - \frac{1}{2\beta}C^{\dagger}_{\alpha} C^{\dagger}_{\alpha'} V_{\alpha \alpha' \beta \beta'} C_{\beta} C_{\beta'} \nonumber\\
       %
       &= \beta\bar{\Delta}_{\alpha \alpha'} V^{-1}_{\alpha \alpha' \beta \beta'} \Delta_{\beta \beta'} - \left( C^{\dagger}_{\gamma} C^{\dagger}_{\gamma'} \Delta_{\gamma \gamma'}+ \bar{\Delta}_{\gamma \gamma'}  C_{\gamma} C_{\gamma'}\right) \nonumber \;.
\end{align}
Where we have used~\cref{inverse relation} to resolve the $V V^{-1}$ and $V^{-1} V V^{-1}$ terms. 
In order to perform the HS-transformation, we must first rewrite the potential 
\begin{equation}
    V_{Qpp'} = \frac{2}{N} \sum_s \frac{g_{\textbf{Q} + \textbf{p'},s}^{(\textbf{p} - \textbf{p'})} g_{\textbf{Q} - \textbf{p'},s}^{(\textbf{p'} - \textbf{p}})}{\left(-i\hbar\Omega_{(p-p')_0} + \hbar \omega_{(\textbf{p} - \textbf{p'})}\right)}\;,
    \label{potential in new coordinates}
\end{equation}
where we have introduced the coordinates $Q = (k + k') / 2 $, $p = p' + q$ and $p' = (k - k')/2$. Using the HS-transformation~\cref{hs-transformation}, we can rewrite the effective action in~\cref{effective action}
\begin{align}
S_{eff}[C^{\dagger},C] &= \sum_{k,\sigma} \epsilon_k C^{\dagger}_{k,\sigma} C_{k,\sigma} -  \frac{1}{2\beta} \sum_{Q,p,p',\sigma,\sigma'} C^{\dagger}_{Q + p,\sigma'} C^{\dagger}_{Q - p,\sigma} V_{Q p p'} C_{Q - p',\sigma}  C_{Q + p',\sigma'}\label{effective action transformed}\\
                       %
                       &=  \sum_{k,\sigma} \epsilon_k C^{\dagger}_{k,\sigma} C_{k,\sigma} +  \sum_{Q,p,p',\sigma,\sigma'} \beta \bar{\Delta}^{Q}_{\sigma' \sigma}(p) V^{-1}_{Q p p'} \Delta^{Q}_{\sigma \sigma'}(p')\nonumber\\
                       %
                       &- \sum_{Q,p,\sigma,\sigma'}\left( C^{\dagger}_{Q + p,\sigma'} C^{\dagger}_{Q - p,\sigma} \Delta^{Q}_{\sigma \sigma'}(p)+ \bar{\Delta}^{Q}_{\sigma' \sigma}(p)  C_{Q - p,\sigma} C_{Q + p,\sigma'} \right)\nonumber\\
                       %
                       &:= \sum_{Q,p,p',\sigma,\sigma'} \beta \bar{\Delta}^{Q}_{\sigma' \sigma}(p) V^{-1}_{Q p p'} \Delta^{Q}_{\sigma \sigma'}(p') + \frac{1}{2} \sum_{Q p} \bar{\Psi}_{Q p} G^{-1}_ {Q p} \Psi_{Q p}\;,
\end{align}
with 
\todo{fortegn på diagonalen?}
\begin{equation}
     G_{Q p}^{-1} = 
\begin{bmatrix}
\epsilon_{Q + p} & 0                & \Delta^{Q}_{\uparrow \uparrow}(p) & \Delta^{Q}_{\uparrow \downarrow}(p) \\
         0       & \epsilon_{Q + p} & \Delta^{Q}_{\downarrow \uparrow}(p) & \Delta^{Q}_{\downarrow \downarrow}(p) \\
         \bar{\Delta}^{Q}_{\uparrow \uparrow}(p)       & \bar{\Delta}^{Q}_{\downarrow \uparrow}(p)                & -\epsilon_{Q-p} & 0 \\
         \bar{\Delta}^{Q}_{\uparrow \downarrow }(p)       & \bar{\Delta}^{Q}_{\downarrow \downarrow}(p)                & 0               & -\epsilon_{Q-p} \\
\end{bmatrix}\;,
\end{equation}
and
\begin{equation}
   \Psi_{Q p} = 
\begin{bmatrix}
     C_{Q + p,\uparrow} \\
     C_{Q + p,\downarrow}  \\
     C^{\dagger}_{Q - p,\uparrow}  \\
     C^{\dagger}_{Q - p,\downarrow}
\end{bmatrix}\;.
\end{equation} 
At this point it is necessary to make the approximation that $Q$ has a fixed value, so that we can remove the sum over $Q$ in~\cref{effective action transformed}. The value of $Q$ is chosen so that it gives the highest possible critical temperature~($T_c$), making it responsible for the phase transition. After fixing $Q$, we can perform the integral  
\begin{align}
    \int D(C^{\dagger},C) e^{\frac{1}{2} \sum_{p} \bar{\Psi}_{p} G^{-1}_ {p} \Psi_{p} / \hbar} &= \Pi_p \sqrt{det{G^{-1}(p)}} \nonumber\\
                                                                                               &= e^{\sum_p \frac{1}{2}\Tr{\ln{G^{-1}(p)}}} \;\nonumber,
\end{align}
where we have used the identity $\ln{\det{A}} = \Tr{\ln{A}}$. The square-root stems from the fact that we are integrating over the Nambu spinors $\bar{\Psi}_{Q p}$ and $\Psi_{Q p}$~\cite{Wegner2016,Hugdal2019} \todo{tror jeg forstår pfaffian ish nå, burde jeg skrive om det eller holder det med referanse?} The path integral~\cref{effective partion function} then becomes
\begin{equation}
    Z_{eff} = \int D(\bar{\Delta}, \Delta) e^{-\left(\beta\sum_{p,p',\sigma,\sigma'} \bar{\Delta}_{\sigma' \sigma}(p) V^{-1}_{p p'} \Delta_{\sigma' \sigma}(p') + \frac{\hbar}{2} \sum_p \Tr{\ln{G^{-1}(p)}}\right) / \hbar}\;. 
\end{equation}
The fermionic fields $C^{\dagger}$ and $C$ have now been replaced with the bosonic fields $\Delta$ and $\bar{\Delta}$.
\section{Mean Field Theory}
\todo{1. Komme til schlawin sin likning 2. Finn relasjon mellom Delta og Deltabar 3. få gap likning med generelle sigmaer 4. jage ned konstant i siste likning, mistener at det kommer fra matsubara sum}
To determine the mean field configuration of $\bar{\Delta}$, we solve the stationary phase equation
\begin{align}
    \frac{\delta S_{eff}}{\delta \Delta_{\sigma_1,\sigma_2}^{p}} &= 0 \nonumber\\
    \beta\sum_{p'} \bar{\Delta}_{\sigma_1 \sigma_2}(p') V^{-1}_{p' p} + \frac{\hbar}{2} \Tr{G(p) 
    \begin{bmatrix}
    0 & 0 & \delta_{\sigma_1 \sigma_2, \uparrow \uparrow} & \delta_{\sigma_1 \sigma_2, \uparrow \downarrow} \\
    0 & 0 & \delta_{\sigma_1 \sigma_2, \downarrow \uparrow} & \delta_{\sigma_1 \sigma_2, \downarrow \downarrow} \\ 
    0 & 0 & 0 & 0 \\ 
    0 & 0 & 0 & 0
    \end{bmatrix}} &=0 \label{stationary phase deltabar}\\
    \beta \bar{\Delta}_{\sigma_1 \sigma_2}(p) + \frac{\hbar}{2} \sum_{p'} \Tr{G(p') 
    \begin{bmatrix}
    0 & 0 & \delta_{\sigma_1 \sigma_2, \uparrow \uparrow} & \delta_{\sigma_1 \sigma_2, \uparrow \downarrow} \\
    0 & 0 & \delta_{\sigma_1 \sigma_2, \downarrow \uparrow} & \delta_{\sigma_1 \sigma_2, \downarrow \downarrow} \\ 
    0 & 0 & 0 & 0 \\ 
    0 & 0 & 0 & 0
    \end{bmatrix}} V_{p' p} &=0 \nonumber;.
\end{align}
The analogus equation for $\Delta$ becomes
\begin{align}
    \beta \Delta_{\sigma_1,\sigma_2}^p + \frac{\hbar}{2} \sum_{p'} V_{pp'} \Tr{G(p') 
    \begin{bmatrix}
    0 & 0 & 0 & 0 \\
    0 & 0 & 0 & 0 \\ 
    \delta_{\sigma_1 \sigma_2, \uparrow \uparrow} & \delta_{\sigma_1 \sigma_2, \uparrow \downarrow} & 0 & 0 \\ 
    \delta_{\sigma_1 \sigma_2, \downarrow \uparrow} & \delta_{\sigma_1 \sigma_2, \downarrow \downarrow} & 0 & 0
    \end{bmatrix}}\;. \label{stationary phase delta}
\end{align}
where the traces are calculated using computer algebra software
\begin{align}
&\Tr{G(p) 
    \begin{bmatrix}
    0 & 0 & \delta_{\sigma_1 \sigma_2, \uparrow \uparrow} & \delta_{\sigma_1 \sigma_2, \uparrow \downarrow} \\
    0 & 0 & \delta_{\sigma_1 \sigma_2, \downarrow \uparrow} & \delta_{\sigma_1 \sigma_2, \downarrow \downarrow} \\ 
    0 & 0 & 0 & 0 \\ 
    0 & 0 & 0 & 0
    \end{bmatrix}}= \\ 
                &+ \frac{\delta_{dd}}{\det{G^{-1}}} \left(\Delta_{uu} \bar{\Delta}_{dd} \bar{\Delta}_{uu} - \Delta_{uu} \bar{\Delta}_{du} \bar{\Delta}_{ud} + \bar{\Delta}_{dd} \epsilon_{Q+p} \epsilon_{Q-p}\right) \nonumber\\ 
                &+ \frac{\delta_{du}}{\det{G^{-1}}} \left(- \Delta_{ud} \bar{\Delta}_{dd} \bar{\Delta}_{uu} + \Delta_{ud} \bar{\Delta}_{du} \bar{\Delta}_{ud} + \bar{\Delta}_{ud} \epsilon_{Q+p} \epsilon_{Q-p}\right) \nonumber\\
                &+ \frac{\delta_{ud}}{\det{G^{-1}}} \left(- \Delta_{du} \bar{\Delta}_{dd} \bar{\Delta}_{uu} + \Delta_{du} \bar{\Delta}_{du} \bar{\Delta}_{ud} + \bar{\Delta}_{du} \epsilon_{Q+p} \epsilon_{Q-p}\right) \nonumber\\
                &+ \frac{\delta_{uu}}{\det{G^{-1}}} \left(\Delta_{dd} \bar{\Delta}_{dd} \bar{\Delta}_{uu} - \Delta_{dd} \bar{\Delta}_{du} \bar{\Delta}_{ud} + \bar{\Delta}_{uu} \epsilon_{Q+p} \epsilon_{Q-p}\right) \;,\nonumber
\end{align}
and
\begin{align}
    &\Tr{G(p) 
    \begin{bmatrix}
    0 & 0 & 0 & 0 \\
    0 & 0 & 0 & 0 \\ 
    \delta_{\sigma_1 \sigma_2, \uparrow \uparrow} & \delta_{\sigma_1 \sigma_2, \uparrow \downarrow} & 0 & 0 \\ 
    \delta_{\sigma_1 \sigma_2, \downarrow \uparrow} & \delta_{\sigma_1 \sigma_2, \downarrow \downarrow} & 0 & 0
    \end{bmatrix}}= \\ 
    &+ \frac{\delta_{dd}}{\det{G^{-1}}} \left(\Delta_{dd} \Delta_{uu} \bar{\Delta}_{uu}  - \Delta_{du} \Delta_{ud} \bar{\Delta}_{uu} + \Delta_{dd} \epsilon_{Q+p} \epsilon_{Q-p}\right)\nonumber \\
    &+ \frac{\delta_{du}}{\det{G^{-1}}} \left(- \Delta_{dd} \Delta_{uu} \bar{\Delta}_{ud} + \Delta_{du} \Delta_{ud} \bar{\Delta}_{ud} + \Delta_{ud} \epsilon_{Q+p} \epsilon_{Q-p}\right)\nonumber \\
    &+ \frac{\delta_{ud}}{\det{G^{-1}}} \left(- \Delta_{dd} \Delta_{uu} \bar{\Delta}_{du} + \Delta_{du} \Delta_{ud} \bar{\Delta}_{du} + \Delta_{du} \epsilon_{Q+p} \epsilon_{Q-p}\right)\nonumber \\
    &+ \frac{\delta_{uu}}{\det{G^{-1}}} \left(\Delta_{dd} \Delta_{uu} \bar{\Delta}_{dd} - \Delta_{du} \Delta_{ud} \bar{\Delta}_{dd} + \Delta_{uu} \epsilon_{Q+p} \epsilon_{Q-p}\right)\;.
\end{align}
We now want to restrict the gap equations to only one of the gaps beeing nonzero. The equations for $\Delta_{uu}$ and $\Delta_{dd}$ triviall

Gives the equations
\begin{align}
     &\beta{\bar{\Delta}}_{uu}^{p} + \frac{\hbar}{2} \sum_{p'} \frac{\bar{\Delta}_{uu}^{p'}}{\Delta_{uu}^{p'} \bar{\Delta}_{uu}^{p'} + \epsilon_{Q+p'} \epsilon_{Q-p'}} V_{p'p}  = 0\nonumber\\
     %
     &\beta \bar{\Delta}_{du}^p + \frac{\hbar}{2} \sum_{p'} \frac{\bar{\Delta}_{ud}^{p'}}{\Delta_{du}^{p'} \bar{\Delta}_{ud}^{p'} + \epsilon_{Q+p'} \epsilon_{Q-p'}}  V_{p'p}  = 0\nonumber\\
     %
     &\beta \bar{\Delta}^{p}_{ud} + \frac{\hbar}{2} \sum_{p'} \frac{\bar{\Delta}_{du}^{p'}}{\Delta_{ud}^{p'} \bar{\Delta}_{du}^{p'} + \epsilon_{Q+p'} \epsilon_{Q-p'}}  V_{p'p} = 0 \nonumber \\
     %
     &\beta{\bar{\Delta}}_{dd}^{p} + \frac{\hbar}{2} \sum_{p'} \frac{\bar{\Delta}_{dd}^{p'}}{\Delta_{dd}^{p'} \bar{\Delta}_{dd}^{p'} + \epsilon_{Q+p'} \epsilon_{Q-p'}}  V_{p'p}  = 0\nonumber \\
     %
     &\beta \Delta_{uu}^p + \frac{\hbar}{2} \sum_{p'} V_{pp'} \frac{\Delta^{p'}_{uu}}{\Delta^{p'}_{uu} \bar{\Delta}^{p'}_{uu} + \epsilon_{Q+p'} \epsilon_{Q-p'}} = 0\nonumber \\
     %
     &\beta \Delta_{du}^p + \frac{\hbar}{2} \sum_{p'} V_{pp'} \frac{\Delta^{p'}_{ud}}{\Delta^{p'}_{ud} \bar{\Delta}^{p'}_{du} + \epsilon_{Q+p'} \epsilon_{Q-p'}} = 0 \nonumber \\
     %
     &\beta \Delta_{ud}^p + \frac{\hbar}{2} \sum_{p'} V_{pp'} \frac{\Delta^{p'}_{du}}{\Delta^{p'}_{du} \bar{\Delta}^{p'}_{ud} + \epsilon_{Q+p'} \epsilon_{Q-p'}} = 0\nonumber \\
     %
     &\beta \Delta_{dd}^p + \frac{\hbar}{2} \sum_{p'} V_{pp'} \frac{\Delta^{p'}_{uu}}{\Delta^{p'}_{uu} \bar{\Delta}^{p'}_{uu} + \epsilon_{Q+p'} \epsilon_{Q-p'}} = 0
\label{gap equations}
\end{align}
where we have used the expression for determinants
\begin{align}
&\det{G^{-1}}_{\Delta_{uu} = \Delta_{dd} = 0} = \left(\Delta_{du} \bar{\Delta}_{ud} + \epsilon_{Q+p} \epsilon_{Q-p}\right) \left(\Delta_{ud} \bar{\Delta}_{du} + \epsilon_{Q+p} \epsilon_{Q-p}\right) \nonumber \\
%
&\det{G^{-1}}_{\Delta_{dd} = \Delta_{ud}  = \Delta_{du} = 0} = \epsilon_{Q+p'} \epsilon_{Q-p'} \left( \Delta^{p'}_{uu} \bar{\Delta}^{p'}_{uu} + \epsilon_{Q+p'} \epsilon_{Q-p'}\right)\label{determinants}\\
%
&\det{G^{-1}}_{\Delta_{uu} = \Delta_{ud}  = \Delta_{du} = 0} = \epsilon_{Q+p'} \epsilon_{Q-p'} \left( \Delta^{p'}_{dd} \bar{\Delta}^{p'}_{dd} + \epsilon_{Q+p'} \epsilon_{Q-p'}\right)\nonumber
\end{align}
We can immediately identify the relation $\bar{\Delta} = \Delta^*$. To see this we use the fact that $V_{pp'}$ is real and that $V_{pp'} = V_{p'p}$ and take the complex conjugate of the equations for $\bar{\Delta}$ in~\cref{gap equations}. Since $\epsilon_{Q + p'}\epsilon_{Q-p'}$ is the only frequency dependent part of the equaion, we can change the direction of summation $p_0 \rightarrow -p_0$ and get $(\epsilon_{Q + p'}\epsilon_{Q-p'})* \rightarrow \epsilon_{Q + p'}\epsilon_{Q-p'}$. Identifying $\bar{\Delta} = \Delta^*$ we get the equations for $\Delta$.

The equations for $\Delta_{ud}$ and $\Delta_{du$ are still coupled to each other. In order to uncouple the equations we introduce the singlet and triplet states $\Delta_a = (\Delta_{ud} - \Delta_{du}) / 2$ og $\Delta_s = (\Delta_{ud} + \Delta_{du}) / 2$, and take only one of the gaps to be nonzero
\begin{align}
\beta \bar{\Delta}^{p}_{ud} + \frac{\hbar}{2} \sum_{p'} \frac{\bar{\Delta}_{du}^{p'}}{\Delta_{ud}^{p'} \bar{\Delta}_{du}^{p'} + \epsilon_{Q+p'} \epsilon_{Q-p'}} V_{p'p} &= 0 \nonumber \\
%
\beta (\bar{\Delta}^{p}_{s} + \bar{\Delta}^{p}_{a}) + \frac{\hbar}{2} \sum_{p'} \frac{\bar{\Delta}_{s}^{p'} - \bar{\Delta}_{a}^{p'}}{(\Delta_s^{p'} + \Delta_a^{p'})(\bar{\Delta}_s^{p'} - \bar{\Delta}_a^{p'})  + \epsilon_{Q+p'} \epsilon_{Q-p'}} V_{p'p} &= 0 \nonumber \\   
%
\rightarrow \beta \bar{\Delta}^{p}_{s} + \frac{\hbar}{2} \sum_{p'} \frac{\bar{\Delta}_{s}^{p'}}{\Delta_s^{p'}\bar{\Delta}_s^{p'}  + \epsilon_{Q+p'} \epsilon_{Q-p'}} V_{p'p} &= 0 \nonumber \\ 
%
\beta \bar{\Delta}^{p}_{a} + \frac{\hbar}{2} \sum_{p'} \frac{- \bar{\Delta}_{a}^{p'}}{-\Delta_a^{p'}\bar{\Delta}_a^{p'}  + \epsilon_{Q+p'} \epsilon_{Q-p'}} V_{p'p} &= 0 \nonumber \;.
\end{align}

Now we restrict ourselves to only one of the gaps beeing nonzero, for simplicity we choose $\Delta_{\uparrow \uparrow}^{p}$. This yields the gap equation
\begin{align}
    \beta{\bar{\Delta}}_{uu}^{p} + \frac{1}{2} \sum_{p'} \frac{\bar{\Delta}_{uu}^{p'}  V_{p'p}}{\Delta_{uu}^{p'} \bar{\Delta}_{uu}^{p'} + \epsilon_{Q+p'} \epsilon_{Q-p'}}  &= 0\nonumber\\
\end{align}
Where we have used $\det{G^{-1}}(p') = \epsilon_{Q+p'} \epsilon_{Q-p'} \left( \Delta^{p'}_{uu} \bar{\Delta}^{p'}_{uu} + \epsilon_{Q+p'} \epsilon_{Q-p'}\right)$

Making the approximation that $\Omega_{\left(p' - p\right)_0} = 0$, the potential~\cref{potential in new coordinates} reduces to
\begin{equation}
    V_{pp'} = \frac{2}{N} \sum_s \frac{g_{\textbf{Q} + \textbf{p'},s}^{(\textbf{p} - \textbf{p'})} g_{\textbf{Q} - \textbf{p'},s}^{(\textbf{p'} - \textbf{p}})}{\hbar \omega_{(\textbf{p} - \textbf{p'})}}\;,
    \label{potential in new coordinates}
\end{equation}


assuming that $\Delta$ is frequency independent, the frequency summation reduces to ($n = p_0'$), $\epsilon_{\textbf{k}} = \epsilon_{\textbf{k}} - \mu$
\todo{legge til del om  $\Omega$ settes lik 0}
\begin{align}
         &=  \sum_{n} \frac{1}{\abs{\Delta}^2 + \epsilon_{Q+p'} \epsilon_{Q-p'}} \nonumber \\
         %
         &=  \sum_{n} \frac{1}{\abs{\Delta}^2 + \left(-i\hbar\omega_n - i \hbar \omega_{Q_0} + \epsilon_{\textbf{Q} + \textbf{p'}}\right) \left(i\hbar\omega_n - i \hbar \omega_{Q_0} + \epsilon_{\textbf{Q} - \textbf{p'}}\right)} \nonumber \\
         %
         &= \sum_{n} \frac{1}{-\hbar^2\left(i\omega_n - E_+^{\textbf{p'}}\right)\left(i\omega_n - E_-^{\textbf{p'}}\right)} \nonumber\\
         %
\end{align}
with 
\begin{equation}
        E_{\pm}^{\textbf{p'}} = \frac{\left(\epsilon_{\textbf{Q} + \textbf{p'}} - \epsilon_{\textbf{Q} - \textbf{p'}}\right) \pm \sqrt{\left(\epsilon_{\textbf{Q} + \textbf{p'}} + \epsilon_{\textbf{Q} - \textbf{p'}}\right)^2 + 4 \abs{\Delta}^2 } }{2\hbar}
\end{equation} 
where we have set $\omega_{Q_0} = 0$ (hvorfor kan vi gjøre dette?).
\begin{align}
         &=  \frac{-\beta / \hbar^2}{E_+^{\textbf{p'}} - E_-^{\textbf{p'}}}\left(\frac{1}{e^{\beta E_+^{\textbf{p'}}} + 1} - \frac{1}{e^{\beta E_-^{\textbf{p'}}} + 1}\right)\nonumber\\
         %
         &=  \frac{-\beta / \hbar^2}{E_+^{\textbf{p'}} - E_-^{\textbf{p'}}}\left(\frac{1}{e^{\beta E_+^{\textbf{p'}}} + 1} + \frac{1}{e^{-\beta E_-^{\textbf{p'}}} + 1} - 1\right)\nonumber\\
         %
         &=  \frac{-\beta / \hbar}{\sqrt{\left(\epsilon_{\textbf{Q} + \textbf{p'}} + \epsilon_{\textbf{Q} - \textbf{p'}}\right)^2 + 4 \abs{\Delta}^2 }}\left(f_{\textbf{p'}} + f_{-\textbf{p'}} - 1\right)\nonumber\\
\end{align}
In the last line we used $\frac{1}{e^X + 1} = 1 - \frac{1}{e^{-X} + 1}$ and $E_-^{\textbf{p}} = E_+^{-\textbf{p}}$. Putting everything together
\begin{equation}
\bar{\Delta}^{p}_{ud} - \frac{1}{2} \sum_{p'} \frac{\bar{\Delta}_{du}^{p'} V_{p'p}}{2\sqrt{\left(\epsilon_{\textbf{Q} + \textbf{p'}} + \epsilon_{\textbf{Q} - \textbf{p'}}\right)^2 / 4 + \abs{\Delta}^2 }}\left(f_{\textbf{p'}} + f_{-\textbf{p'}} - 1\right) &= 0     
\end{equation}

\textbf{schlawin side 3 triplet and singlet order?}
\\
\\
\textbf{virker som om bar til komplekskonjugert er riktig i alle tilfellene?}
\\
\\
\textbf{Ser ut nå som det bare mangler en faktor 2}
\\
\\
\textbf{Burde vel skrive om valg av Q og amperian superconductivity?(mer et spørsmål om hvor utfyllende jeg skal skrive nå)}
\\
\\
\textbf{Burde man ikke kunne bruke at interaksjon egt er uavhengi av spin?}
\\
\\
\textbf{Tempere s 389, forstår ikke hva han sier om integration measure. Virker ikke som dette er et problem i Atland?}
\printbibliography





\end{document}
